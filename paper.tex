\documentclass{llncs}

\usepackage{url}
\usepackage{xspace}

\newcommand{\OpenTheory}{OpenTheory\xspace}
\newcommand{\secref}[1]{Section~\ref{sec:#1}}

\begin{document}

\title{Standalone Tactics using OpenTheory}

\author{Ramana Kumar\thanks{supported by the Gates Cambridge Trust}\inst{1}\and Joe Hurd\inst{2}}

\institute{\email{Ramana.Kumar@cl.cam.ac.uk}\and\email{joe@galois.com}}

\maketitle

\begin{abstract}
Tactics, functions that compute low-level inferences to achieve a high-level reasoning goal, are usually developed within and tied to a particular proof assistant, which leads to code duplication across different systems and inhibits reuse.
As formal methods are applied to larger problems, projects spanning multiple proof assistants benefit more from interoperability and access to the union of all implemented tactics.
Using the OpenTheory proof exchange format, we show how to turn a tactic implemented for one system into one available to many via the web.
This enables, for example, LCF-style proof reconstruction efforts for fast automatic provers to be shared by users of multiple proof assistants and removes the need for each user to install the automatic prover.
\end{abstract}

\section{Introduction}
There are many systems for interactively developing machine-checked formal theories, including HOL4~\cite{slind08brief}, HOL Light~\cite{DBLP:conf/tphol/Harrison09a}, ProofPower~\cite{ProofPower}, Isabelle/HOL~\cite{wenzel08isabelle}, and Coq~\cite{bertot08short}.
%All but the last are based on higher-order logic~\cite{church40formulation,gordon94hol} and follow the LCF tradition~\cite{gordon79lcf} wherein theorems are represented by an abstract datatype all of whose constructors, representing inference rules, are contained in a small, trusted kernel.
%In LCF-style provers, powerful proof procedures are kept trustworthy by, ultimately, only calling kernel functions to create theorems.
The logic implemented by these systems (except Coq) is essentially the same, but the collections of libraries and tactics built atop the logical kernels differ.
Where similar functionality exists in two systems it is usually the result of duplicate effort.
For instance, Kumar and Weber~\cite{DBLP:conf/itp/KumarW11} and Kun\v{c}ar~\cite{DBLP:conf/itp/Kuncar11} give independent integrations of a quantified boolean formula solver into a higher order logic based system (HOL4 in one case and HOL Light in the other).

We contend that tactics or libraries for proof assistants need only be written and maintained in one place rather than once per system. 
An added advantage when the tactic is an integration of an external tool is that a user of the proof assistant need not also have installed the external tool, because a standalone tactic can be accessed from a web server where the tool is installed.
The contributions of this paper are:
\begin{itemize}
\item
A tested methodology (\secref{opentheory}) for making existing libraries standalone, using the \OpenTheory~\cite{hurd2009} article format;
\item
A variety of standalone tactics ready for use as web services (\secref{web}); and,
\item
Performance data (\secref{performance}) comparing standalone tactics against direct implementations.
\end{itemize}

\section{Tactics in Proof Assistants}

Classification and examples of the kinds of libraries and tactics implemented ontop of the kernel in various systems.

Certain kinds of functionality, or improvements in functionality (e.g. Isabelle function package over TFL) become restricted to a single proof assistant unnecessarily.

\section{\OpenTheory for Tactic Communication}
\label{sec:opentheory}

Introduction to article format.
Representing terms and proofs in the format.
Viewing tactics as derived inference rules.
Implementation details to make standalone.

\section{Making Existing Tactics into Web Services}
\label{sec:web}

Implementation details to put on the web.
Examples of tactics for which it is done: HolQbf, HolSmt, HolSat, sledgehammer, TFL, function package, Metis, datatype packages, arithmetic decision procedure.

\section{Performance}
\label{sec:performance}

Speed comparisons against using built-in tactics.
Amount of code and time required, compared to porting tactics at the implementation level.

\section{Related Work}

Evidential tool bus.
System on TPTP.
Natural extension of the integration of automatic tools into interactive systems.

\section{Conclusions}

Additional (non-proof) work tactics might also do needs to be represented on another channel.

\bibliographystyle{splncs}
\bibliography{paper}

\end{document}
